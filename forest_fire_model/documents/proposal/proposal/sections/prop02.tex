\section{Forest Fire Model}
This paper \cite{inproceedings} presents a cellular automata approach to forest fire simulation that can be enhanced with particle methods. While traditional CA models use grid cells, a particle-based representation allows for more realistic fire dynamics and environmental interactions.

\subsection{Implementation Plan}
\begin{enumerate}
    \item \textbf{Basic CA-Particle Hybrid Model}
    \begin{itemize}
        \item Implement a 2D grid where each cell can be in states like ``fuel'' (unburned), ``burning'', or ``burned''
        \item Create a particle system where fire is represented by discrete particles with:
        \begin{itemize}
            \item Position and velocity vectors
            \item Temperature/intensity attributes
            \item Lifetime properties
        \end{itemize}
        \item Implement state transition rules:
        \begin{itemize}
            \item A burning cell generates fire particles
            \item Fire particles propagate based on environmental factors
            \item Fuel cells ignite based on proximity to fire particles and probabilistic rules
            \item Burned cells remain in a non-flammable state
        \end{itemize}
    \end{itemize}

    \item \textbf{Environmental Factors Integration}
    \begin{itemize}
        \item Implement wind as a vector field affecting particle movement
        \item Model terrain topography influencing fire spread (uphill spread accelerates)
        \item Add fuel heterogeneity with different burn probabilities and intensities
        \item Incorporate moisture content affecting ignition probability
    \end{itemize}

    \item \textbf{Validation and Analysis}
    \begin{itemize}
        \item Measure fire front propagation rates under different conditions
        \item Analyze emergent fire patterns and shapes
    \end{itemize}
\end{enumerate}
