\section{Methodology}
The simulation uses a hybrid approach by combining a 2D CA with a dynamic particle system. The CA grid represents the landscape, with each cell characterized by its state (burning, burned etc) and environmental properties (fuel type, moisture, elevation). Fire is propagated and represented by discrete fire particles. These particles are generated by burning cells and move across the grid influence by environmental factors and can ignite susceptible fuel cells.

\subsection{Cell states}
Each cell in the simulation grid can exist in one of the following states
\begin{itemize}
	\item \textit{FUEL}: Represents unburned vegetation or combustible material.
	\item \textit{BURNING}: Represents the cell being currently on fire.
	\item \textit{BURNED}: Represents a cell where the fuel has been fully consumed and is no longer flammable.
	\item \textit{EMPTY}: Represents a cell with no fuel, acting as natural firebreaks (such as roads, water bodies etc).
\end{itemize}
\subsection{Fire Particle System}
Each Fire particle has the following attributes
\begin{itemize}
	\item \textit{Position}: 2D vector, representing the location on the grid.
	\item \textit{Velocity}: 2D vector, determining  the movement per time step.
	\item \textit{Intensity}: Scalar value representing the heat/energy of the particle, which influence the ability to ignite fuel and its own behavior.
	\item \textit{Lifetime}: The maximum duration (in time steps) the particle can exist.
	\item \textit{Age}: The current duration the particle has existed.
\end{itemize}
these properties are effected as follows at each time step
\begin{enumerate}
	\item \textbf{Aging}: Particle age is incremented and if it exceeds the its lifetime or the intensity drops below a threshold, particle becomes inactive.
	\item \textbf{Wind}: The local wind vector contributes to the particle's velocity.
	\item \textbf{Terrain}: The local terrain gradient is calculated. Based on that the particle experiences an acceleration uphill with the effect being increased for steeper slopes, which effects the velocity.
	\item \textbf{Random Movement}: A small random component is added to the velocity, scaled by the particles intensity to simulate erratic ember movement.
	\item \textbf{Position Update}: Position is updated based on the velocity and the time step.
	\item \textbf{Boundary Conditions}: If particle moves out of the grid boundaries, its intensity is set to zero, leading to the particle becoming inactive.
	\item \textbf{Intensity Decay}: The particle intensity decays over time, with decay rate potentially increasing over time.
\end{enumerate}

\subsection{Fire Spread and Ignition}
\subsubsection{Ignition}
Active fire particles can ignite nearby \textit{FUEL} cells. For each active particle:
\begin{itemize}
	\item An ignition radius based on the intensity is determined.
	\item Cell within this radius are considered for ignition.
	\item The probability of a \textit{FUEL} cell being ignite is computed based on:
	      \begin{enumerate}
		      \item Higher intensity increases probability.
		      \item Closer cells have a higher probability.
		      \item Higher fuel density value increases probability.
		      \item Lower moisture content increases probability.
	      \end{enumerate}
	\item A random number is used in conjunction with the computed probability to determine if ignition occurs.
\end{itemize}
\subsubsection{Cell State Transitions}
The Cell states transitions are based on the following rules:
\begin{itemize}
	\item \textbf{FUEL to BURNING}: A cell transitions successfully from FUEL to BURNING, if successfully ignited by a fire particle. When a Cell is ignited, new particle spawn at its location, the number of new particles and their properties is determined with some randomness.
	\item \textbf{BURNING to BURNED}: A burning cell has a fixed probability of transitioning to the BURNED state, simulating fuel consumption.
	\item \textbf{Particle Generation from Burning Cells}: Each burning cell have a small probability generating new fire particle, simulating ongoing ember production from an active fire.
\end{itemize}
\subsection{Environmental Factors}
\begin{itemize}
	\item \textbf{Wind Model}: Wind is modeled by a 2D vector field, where each cell has an wind vector. The wind can be initialized uniformly or variably (base direction/strength with smoothed random variations).
	\item \textbf{Terrain Model}: Terrain is represented as a 2D scalar field, where each cells stores an elevation. The terrain is either generated by a random height map and then smoothing it using a gaussian or can be setup manually.
	\item \textbf{Fuel Heterogeneity}: Fuel is modeled by a 2D scalar field, where each cell has a value representing its fuel type and density, where a higher density inidicates more flammable fuel.
	\item \textbf{Moisture Content}: The Moisture content is modeled by a 2D scalar field, where each cell has a value between 0 and 1, that represents the moisture level. These levels are smoothed by gaussian filter and higher moisture content reduces the likelihood of ignition.
\end{itemize}

\section{Scenarios}

\subsection{Baseline Scenario}
\subsection{Wind Effects Scenarios}
\subsection{Terrain Effects Scenarios}
\subsection{Fuel and Moisture Scenarios}
\subsection{Combined Factor Scenarios (Interaction Effects)}
