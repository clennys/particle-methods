\section{Methodology}
The simulation uses a hybrid approach by combining a 2D CA with a dynamic particle system. The CA grid represents the landscape, with each cell characterized by its state (burning, burned etc) and environmental properties (fuel type, moisture, elevation). Fire is propagated and represented by discrete fire particles. These particles are generated by burning cells and move across the grid influence by environmental factors and can ignite susceptible fuel cells.

\subsection{Cellular Automata Framework}
Each cell in the simulation grid can exist in one of the following states
\begin{itemize}
	\item \textit{FUEL}: Represents unburned vegetation or combustible material.
	\item \textit{BURNING}: Represents the cell being currently on fire.
	\item \textit{BURNED}: Represents a cell where the fuel has been fully consumed and is no longer flammable.
	\item \textit{EMPTY}: Represents a cell with no fuel, acting as natural firebreaks (such as roads, water bodies etc).
\end{itemize}
\subsection{Fire Particle System}
Each Fire particle has the following attributes, representing the physical transport of heat, embers and energy across the simulation landscape:
\begin{itemize}
	\item \textit{Position} $(x, y)$: 2D vector, representing the location on the grid.
	\item \textit{Velocity} $(v_x, v_y)$: 2D vector, determining  the movement per time step.
	\item \textit{Intensity} $(I)$: Scalar value representing the heat/energy of the particle, which influence the ability to ignite fuel and its own behavior.
	\item \textit{Lifetime} $(L)$: The maximum duration (in time steps) the particle can exist.
	\item \textit{Age} $A$: The current duration the particle has existed.
\end{itemize}

\subsubsection{Particle Dynamics}
At each timestep, the particles properties are updated using the following processes:
\begin{enumerate}
	\item \textbf{Aging Process}: Particle age is incremented and if it exceeds its lifetime or the intensity drops below a threshold, the particle becomes inactive.
	\item \textbf{Wind Influence}: The local wind vector contributes to the particle's velocity as follows
	      \begin{equation}
		      \vec{v}_{\text{new}} = \vec{v}_{\text{old}} + \vec{w}_{\text{local}} \times dt \times (1 + \varepsilon)
		      \label{eq:wind}
	      \end{equation}
	\item \textbf{Terrain Effects}: The local terrain gradient is calculated as follows
	      \begin{equation}
		      a_{\text{terrain}} = \nabla h \times k_{\text{slope}} \times \left (1 + 2 \vert \nabla h \vert \right)
		      \label{eq:terrain}
	      \end{equation}
	      where $\nabla h$ is the terrain gradient, $k$ the slope effect coefficient and the the final ter amplifies the effect for steeper slopes.
	\item \textbf{Random Movement}: A small random component is added to the velocity, scaled by the particles intensity to simulate erratic ember movement.
	      \begin{equation}
		      \vec{v}_{\text{random}} = \hat{r} \times k_{\text{random}} \times I \times dt
		      \label{eq:rand_mov}
	      \end{equation}
	\item \textbf{Position Update}: Position is updated using the explicit euler integration.
	      \begin{equation}
		      \vec{x}_{\text{new}} = \vec{x}_{\text{old}} + \vec{v} \times dt
		      \label{eq:euler}
	      \end{equation}
	\item \textbf{Boundary Conditions}: If particle moves out of the grid boundaries, its intensity is set to zero, leading to the particle becoming inactive.
	\item \textbf{Intensity Decay}: The particle intensity decays over time, with decay rate potentially increasing over time.
	      \begin{equation}
		      I_{\text{new}} = I_{\text{old}} \times \left(k_{\text{decay}} - 0.05 \times \min\left(1, \frac{A}{L}\right)\right)
		      \label{eq:decay}
	      \end{equation}
	      where $k_{decay}$ is the base decay rate and the age factor increases decay for older particles.

\end{enumerate}

\subsection{Fire Spread and Ignition}
\subsubsection{Ignition}
Active fire particles can ignite nearby \textit{FUEL} cells, using the following probability:
\begin{equation}
	P_{\text{ignite}} = I_{\text{particle}} \times (1 - \frac{d}{r_{\text{max}}}) \times F_{\text{fuel}} \times (1 - M) \times P_{\text{base}}
	\label{eq:fire_ign}
\end{equation}
where
\begin{itemize}
	\item \textit{Intensity ($I$):} Fire heat/energy
	\item \textit{Distance ($d/r_{\text{max}}$):} Proximity decay effect, where $d$ is the distance between the fuel patcha and the particle and $r_{\text{max}}$ is the maximum ignition radius.
	\item \textit{Fuel ($F$):} Vegetation flammability
	\item \textit{Moisture ($M$):} Resistance factor
\end{itemize}

For each active particle:
\begin{itemize}
	\item An ignition radius based on the intensity is determined.
	\item Cell within this radius are considered for ignition.
	\item The probability of a \textit{FUEL} cell being ignite is computed based on:
	      \begin{enumerate}
		      \item Higher intensity increases probability.
		      \item Closer cells have a higher probability.
		      \item Higher fuel density value increases probability.
		      \item Lower moisture content increases probability.
	      \end{enumerate}
	\item A random number is used in conjunction with the computed probability to determine if ignition occurs.
\end{itemize}
\subsubsection{Cell State Transitions}
The Cell states transitions are based on the following rules:
\begin{itemize}
	\item \textbf{FUEL to BURNING}: A cell transitions successfully from FUEL to BURNING, if successfully ignited by a fire particle. When a Cell is ignited, new particle spawn at its location, the number of new particles and their properties is determined with some randomness.
	\item \textbf{BURNING to BURNED}: A burning cell has a fixed probability of transitioning to the BURNED state, simulating fuel consumption.
	\item \textbf{Particle Generation from Burning Cells}: Each burning cell have a small probability generating new fire particle, simulating ongoing ember production from an active fire.
\end{itemize}
\subsection{Environmental Factors}
\begin{itemize}
	\item \textbf{Wind Model}: Wind is modeled by a 2D vector field, where each cell has an wind vector. The wind can be initialized uniformly or variably (base direction/strength with smoothed random variations).
	\item \textbf{Terrain Model}: Terrain is represented as a 2D scalar field, where each cells stores an elevation. The terrain is either generated by a random height map and then smoothing it using a gaussian or can be setup manually.
	\item \textbf{Fuel Heterogeneity}: Fuel is modeled by a 2D scalar field, where each cell has a value representing its fuel type and density, where a higher density inidicates more flammable fuel

	      \begin{itemize}
		      \item Grassland (0.4-0.6): Low flammability
		      \item Light Forest (0.7-0.9): Moderate flammability
		      \item Mixed Forest (1.0-1.2):Standard flammability reference case
		      \item Dense Forest (1.3-1.7): High flammability
		      \item  Dry Brush (1.8-2.2): Very high flammability
	      \end{itemize}
	\item \textbf{Moisture Content}: The Moisture content is modeled by a 2D scalar field, where each cell has a value between 0 and 1, that represents the moisture level. These levels are smoothed by gaussian filter and higher moisture content reduces the likelihood of ignition.
\end{itemize}

\subsection{Scenario-Specific Map Generation}
The simulation supports multiple map types, each with unique features of interest and characteristics:
\begin{itemize}
	\item \textbf{Forest Map}: A pure fuel environment with heterogenous fuel types and natural moisture variations.
	\item \textbf{Wildland Urban Interface (WUI) Map}: The goal is to simulate an environement that includes flammable wildland and human structures such as buildings, furthermore around these structures the moisture level is increased.
	\item \textbf{Coastal Map}: The map intends to simulate a realistic coastline with different fuel types based on the distance to the coast. It also includes seasonal waterways and wind with variable strength blowing from the cooler sea inland.
\end{itemize}
\subsection{Implementation Details}
The simulation was implemented in Python and includes a real time visualization and key data collection. Furthermore the data can then be post processed with an additional script which generates all the plots (and more), which can be seen in this report. For a more detailed description on the structure of the implementation and how to run the simulation consult the \texttt{README.md} file in the root folder of the project.
