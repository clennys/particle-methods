\section{Conclusion}
This particle-based forest fire simulation successfully demonstrates enhanced physical realism compared to traditional cellular automata approaches. The hybrid model effectively captures complex fire dynamics including ember transport, environmental interactions, and realistic fire spread patterns across diverse scenarios.\newline
\newline
The three scenarios demonstrate distinctly different fire behaviors, validating the model's capability to capture diverse fire dynamics:
\begin{itemize}
	\item \textbf{Environmental Factor Effectiveness:} Moisture content emerges as the most effective natural fire suppressant, with coastal moisture gradients reducing burned area by 57.5\% compared to drought conditions. Wind effects create predictable spread patterns amenable to strategic fire management, while fuel management around structures reduces local ignition probability by approximately 70\%.
	\item \textbf{Model Validation}: The simulation successfully reproduces key fire behavior phenomena including exponential growth phases, wind-driven fire elongation, spotting through ember transport, fuel depletion effects, and barrier effectiveness in containing fire spread.
\end{itemize}
Future enhancements could include 3D fire behavior with canopy dynamics, detailed combustion chemistry, suppression agent modeling, and dynamic weather patterns. The particle-based approach demonstrates significant potential for advancing wildfire modeling capabilities and supporting evidence-based fire management decisions in increasingly fire-prone landscapes.
