\section{Conclusion}
This hybrad forest fire simulation project, showed successfully how to enhance physical the CA model with realism using a particle based approach. The model effectively captures a more complex fire dynamics, this includes ember transport, environmental interactions, and realistic fire spread patterns across diverse scenarios.\newline
The three scenarios demonstrate distinctly different fire behaviors, validating the model's capability to capture diverse fire dynamics:
\begin{itemize}
	\item \textbf{Environmental Factor Effectiveness:} Moisture content and physical barriers emerged as the most effective natural fire suppressant, with coastal moisture gradients reducing burned area by 57.5\% compared to drought conditions. Wind effects create the expected spread patterns, which could be used for strategic fire management.
	\item \textbf{Model Validation}: The simulation successfully captured and reproduced key fire behavior phenomena including exponential growth phases, wind-driven fire elongation, ember transport, fuel depletion effects, and barrier effectiveness in containing fire spread.
\end{itemize}
The next natural step in enhancing this simulation would be using data from a real past wildfire occurence and do a comparitive analysis. Further enhancments could include 3D fire behavior with canopy dynamics, combustion chemistry, suppression agent modeling and more complex dynamic weather patterns.\newline
The particle-based approach demonstrates real potential for enhacing wildfire modeling capabilities and supporting evidence-based fire management decisions in fire-prone landscapes.
