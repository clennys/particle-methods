\section{Boids Flocking Algorithm}
I propose to implement Reynolds' original Boids algorithm \cite{reynolds_flocks_1987}, which simulates the flocking behavior of birds through three simple steering behaviors:
%
\begin{itemize}
	\item \textit{Separation}: Avoid crowding nearby flockmates
	\item \textit{Alignment}: Steer towards the average heading of nearby flockmates
	\item \textit{Cohesion}: Steer towards the average position of nearby flockmates
\end{itemize}%
%
Each boid will be modeled as a particle as follows:
\begin{itemize}
	\item Position, velocity and acceleration vector.
	\item Max. Speed and steering force limitations.
	\item A radius for defining which other boids it can "see".
\end{itemize}
For each time step of the simulation the following steps would be computed:
\begin{enumerate}
	\item Compute seperation, alignment and cohesion
	\item Apply weights to each force
	\item Update velocity and position using time integration.
	\item Handling periodic boundary condition.
	\item Render simulation state.
\end{enumerate}
the following parameters could than be systematically change to explore the systems behavior:
\begin{itemize}
	\item Neighborhood radius
	\item Weights of forces
	\item Flock size
\end{itemize}
\subsection{Implementation Plan}
\begin{enumerate}
	\item \textbf{Basic Model Implementation}
	      \begin{itemize}
		      \item Create a particle class with position, velocity, acceleration vectors, and perception radius
		      \item Implement the three core steering behaviors:
		            \begin{itemize}
			            \item Separation: Avoid crowding nearby flockmates
			            \item Alignment: Steer towards the average heading of nearby flockmates
			            \item Cohesion: Steer towards the average position of nearby flockmates
		            \end{itemize}
		      \item Apply force limitations (max. speed and steering force)
		      \item Implement periodic boundary conditions for continuous movement
		      \item Develop a visualization system to render boids with directional indicators
	      \end{itemize}

	\item \textbf{Parameter Exploration}
	      \begin{itemize}
		      \item Systematically vary and document the effects of:
		            \begin{itemize}
			            \item Neighborhood radius (perception distance)
			            \item Relative weights of each steering force
			            \item Flock size and density
			            \item Time step size for numerical integration
		            \end{itemize}
		      \item Record emergent patterns and phase transitions between different flocking behaviors
	      \end{itemize}
	\item \textbf{Extensions and Analysis}\\
	      These are possible extensions, where I would implement a subset of them:
	      \begin{itemize}
		      \item \textbf{Obstacle Avoidance and Environmental Factors}
		            \begin{itemize}
			            \item Add obstacle avoidance steering behavior
			            \item Add wind or similar effects that influence movement
			            \item Create a food source that attracts the flock
		            \end{itemize}
		      \item \textbf{Leadership Dynamics}
		            \begin{itemize}
			            \item Implement a ``follow the leader'' steering behavior
			            \item Explore different types of leadership (fixed, emergent, rotating)
		            \end{itemize}
		      \item \textbf{Predator-Prey Dynamics}
		            \begin{itemize}
			            \item Implement a predator with different movement patterns
			            \item Give the flock a mechanism to avoid the predator
			            \item Analyze how flocking provides protection against predation
		            \end{itemize}
		      \item \textbf{Advanced Perception Model}
		            \begin{itemize}
			            \item Replace simple radius-based perception with vision cones
		            \end{itemize}
	      \end{itemize}
\end{enumerate}
